\chapter{Sistemas dinâmicos}
\label{cap:dinamicos}

\begin{definicao}
   Um sistema dinâmico é uma tripla ordenada
   $(\mathbb{T}, \mathcal{E}, \Phi)$
   onde $\mathbb{T}$ é um monoide aditivo,
   $\mathcal{E}$ é um conjunto não vazio e
   $\Phi$ é uma função
   $$
      \Phi: I \times \mathcal{E} \rightarrow \mathcal{E},
   $$
   com $I \subset \mathbb{T}$, tal que
   \begin{itemize}
      \item $\Phi(0, x) = x$,
      \item $\Phi(t_1, \Phi(t_0, x)) = \Phi(t_1 + t_0, x)$,
   \end{itemize}
   para todos $x\in\mathcal{E}$, $(t_0, t_1)\in I \times I$.
\end{definicao}

O conjunto $\mathcal{E}$ é comumente chamado de espaço de estados
enquanto que a função $\Phi$ é chamada de função de evolução.
Note que a função de evolução é uma ação de monoide sobre espaço de estados.

Em muitos casos a função de evolução é, \textit{a priori},
desconhecida.
Entretanto, nesses casos a função de evolução deve ser obtida a partir
estruturas algébricas ou geométricas adicionadas ao modelo.
Por exemplo, se $\mathbb{T} = \mathbb{R}$ e se
um atlas diferenciável mapeia o espaço de estados $\mathcal{E}$
então pode-se buscar por um campo vetorial
$X:\mathcal{E} \rightarrow T(\mathcal{E})$
tal que a função de evolução seja o seu escoo, definindo uma
equação da forma

\begin{equation}
   \frac{d}{dt} \Phi_x(t) = X \circ \Phi_x(t),
\end{equation}

\noindent
onde $\Phi_x: t \mapsto \Phi(t, x)$.
Um outro exemplo (que pode ser interpretado como a expressão local do anterior),
é quando $\mathcal{E}$ possui uma estrutura vetorial,
nesse caso pode-se buscar um operador
$\hat{D}: \mathcal{E} \rightarrow \mathcal{E}$
tal que

\begin{equation}
   \frac{d}{dt} \Phi_x(t) = \hat{D} \circ \Phi_x(t).
\end{equation}

\noindent
Essas equações são chamadas de \textit{equações diferenciais ordinárias},
o problema de encontrar a função de evolução a partir de tais equações
se chama \textit{problema de valor inicial},
pois está sujeito à condição de que $\Phi_x(0) = x$,
onde $x$ é um dado estado inicial.
