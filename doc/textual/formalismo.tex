\chapter{Formalismo matemático}

\section{Equações de onda}

Ao analisar equações como as do oscilador harmônico,
as quanções de onda numa corda, ondas císmicas, as equações de Maxwell,
a equação de Schrödinger, dentre outras,
podemos notar similaridades nessas equações que nos levam a descrever
de uma maneira mais geral a álgebra das equações de ondas.

O modelo físico para fenômenos ondulatórios onde se observa
o princípio da superposição consiste de um sistema dinâmico
$(\mathbb{R}, \mathcal{E}, \Phi)$
-- ver apêndice \ref{cap:dinamicos} --,
onde o espaço de estados $\mathcal{E}$ é um espaço vetorial%
\footnote{Necessariamente topológico e localmente convexo},
escalares podem ser tanto números reais como complexos,
e a função de evolução $\Phi$ pode ser inferida a partir
de um operador linear
$\hat{D}:\mathcal{E}\rightarrow\mathcal{E}$ através da equação

\begin{equation}\label{eq:EDO}
   \frac{d}{dt} \Phi_{x_0}(t) = \hat{D} \circ \Phi_{x_0}(t) + x_s,
\end{equation}

\noindent
onde ${x_0}\in\mathcal{E}$ é o estado inicial do ondulante
e $x_s\in\mathcal{E}$ é um termo de fonte
que, para o interesse do presente trabalho, será sempre o vetor nulo.
O operador linear $\hat{D}$ deve ser
antihermitiano com respeito a algum produto interno, isto é,
se $\langle\cdot,\cdot\rangle$ denota tal produto interno, então deve-se ter

\begin{equation}
   \langle \hat{D}(x), y\rangle = -\langle x, \hat{D}(y)\rangle.
\end{equation}

Dessa simples estrutura vêm propriedades físicas muito relevantes
como a natureza oscilatória do ondulante e a conservação de energia.
Os seguintes teoremas enunciam tais propriedades de maneira mais formal.

\begin{teorema}
   Se o espaço de estados $\mathcal{E}$
   é um espaço de Hilbert com respeito a algum produto interno
   $\langle\cdot,\cdot\rangle$
   e a função de evolução satisfaz à equação \ref{eq:EDO}
   para algum operador linear antihermitiano $\hat{D}$
   então
   $$
      \frac{d}{dt} \langle \Phi_{x_0}(t), \Phi_{x_0}(t) \rangle = 0.
   $$
\end{teorema}

\begin{prova}
   Basta calcular:
   $$\begin{aligned}
      \frac{d}{dt} \langle \Phi_{x_0}(t), \Phi_{x_0}(t) \rangle
      &= \left\langle \frac{d}{dt} \Phi_{x_0}(t), \Phi_{x_0}(t) \right\rangle +
         \left\langle \Phi_{x_0}(t), \frac{d}{dt} \Phi_{x_0}(t) \right\rangle \\
      &= \left\langle \hat{D} \circ \Phi_{x_0}(t), \Phi_{x_0}(t) \right\rangle +
         \left\langle \Phi_{x_0}(t), \hat{D} \circ \Phi_{x_0}(t) \right\rangle \\
      &= \left\langle \hat{D} \circ \Phi_{x_0}(t), \Phi_{x_0}(t) \right\rangle -
         \left\langle \hat{D} \circ \Phi_{x_0}(t), \Phi_{x_0}(t) \right\rangle \\
      &= 0 \qedhere
   \end{aligned}$$
\end{prova}

\begin{teorema}
   \textit{a)}
      Se o corpo for o conjunto dos reais, $\mathbb{R}$,
      então todo autovalor de $\hat{D}$ é nulo.
   \textit{b)}
      Se o corpo de $\mathcal{E}$ for o conjunto dos complexos, $\mathbb{C}$,
      então todo autovalor de $\hat{D}$ possui parte real nula,
      isto é, é puramente imaginário.
\end{teorema}

\begin{prova}
   Suponha que $\lambda$ é um autovalor de $\hat{D}$ associado ao autovetor
   unitário $\varphi$, então
   $
      {\lambda} =
      {\langle \hat{D}(\varphi), \varphi \rangle} =
      {- \langle \varphi, \hat{D}(\varphi) \rangle}
   $:
   \textit{a)}
      Por simetria do produto interno
      $
         {\langle \varphi, \hat{D}(\varphi) \rangle} =
         {\langle \hat{D}(\varphi), \varphi \rangle}
      $
      e portanto $\lambda = -\lambda$,
      que somente é possível caso $\lambda$ seja nulo;
   \textit{b)}
      Pela simetria conjungada do produto interno
      $
         {\langle \varphi, \hat{D}(\varphi) \rangle} =
         \overline{\langle \hat{D}(\varphi), \varphi \rangle}
      $
      e portanto $\lambda = - \bar{\lambda}$
      que somente é possível caso a parte real de $\lambda$ seja nula.
\end{prova}

\begin{exemplo}
   A segunda lei de Newton para dois corpos ligados por uma mola
   fornece o seguinte modelo:
   Considere $\mathcal{E} = \mathbb{R}^4$,
   onde um estado $(q^0, q^1, p_0, p_1)\in\mathcal{E}$
   informa os deslocamentos $q^0$ e $q^1$ de cada corpo com respeito
   à posição na qual a mola está relaxada e tammbém os momentos lineares
   $p_0$ e $p_1$ na direção da linha reta imaginária que une os dois corpos,
   a segunda lei de Newton é então escrita como o seguinte sistema de equações
   $$
      \left\{\begin{aligned}
         \dot q^0(t) &= \frac{1}{2 m_0} p_0(t), \\
         \dot q^1(t) &= \frac{1}{2 m_1} p_1(t), \\
         \dot p_0(t) &= \kappa \cdot q^1(t) - \kappa \cdot q^0(t), \\
         \dot p_1(t) &= \kappa \cdot q^0(t) - \kappa \cdot q^1(t),
      \end{aligned}\right.
   $$
   onde $m_0v\in\mathbb{R}$ e $m_1\in\mathbb{R}$ representam as massas
   de cada corpo.
   Note que
   $\hat{D}:(q^0, q^1, p_0, p_1)\mapsto\left(
      \frac{1}{2 m_0} p_0(t),
      \frac{1}{2 m_1} p_1(t),
      \kappa \cdot q^1(t) - \kappa \cdot q^0(t),
      \kappa \cdot q^0(t) - \kappa \cdot q^1(t)
   \right)$
   é simétrico com respeito ao produto interno
   $$
      \langle(q^0, q^1, p_0, p_1), (Q^0, Q^1, P_0, P_1)\rangle\mapsto
      \frac{1}{m_0} p_0 P_0 + \frac{1}{m_1} p_1 P_1 +
      \kappa \cdot (q^1 - q^0) \cdot (Q^1 - Q^0).
   $$
\end{exemplo}

Lorem ipsum

\begin{exemplo}
   Se dois átomos contribuem cada um com um orbital para ser ocupado por
   um elétron então pode modelar o elétron da seguinte maneira:
   Considere $\mathcal{E} = \mathbb{C}^2$, onde cada estado
   $(z^0, z^1)\in\mathcal{E}$ informa as probabilidades
   $|z^0|^2$ e $|z^1|^2$ de encontrar o elétron no orbital
   do primeiro ou do segundo átomo,
   no modelo tight binding a evolução do estado satisfaz o sistema de equações:
   $$
      \left\{\begin{aligned}
         i \hbar \, \dot z^0(t) &= \epsilon_0 \, z^0(t) + \tau \, z^1(t), \\
         i \hbar \, \dot z^1(t) &= \tau \, z^0(t) + \epsilon_1 \, z^1(t),
      \end{aligned}\right.
   $$
   onde $\epsilon_0\in\mathbb{R}$ e $\epsilon_1\in\mathbb{R}$
   representam a energia de ligação elétron em cada orbital e
   $\tau\in\mathbb{R}$ é a taxa de transição entre orbitais.
   O operador $\hat{D}:(z^0, z^1)\mapsto\left(
      \frac{\epsilon_0}{i \hbar} z^0 + \frac{\tau}{i \hbar} z^1,
      \frac{\tau}{i \hbar} z^0 + \frac{\epsilon_1}{i \hbar} z^1
   \right)$ é claramente antihermitiano com respeito ao produto interno usual,
   $\langle\cdot,\cdot\rangle_{\mathbb{C}^2}$.
\end{exemplo}

\section{Formalismo hamiltoniano}

